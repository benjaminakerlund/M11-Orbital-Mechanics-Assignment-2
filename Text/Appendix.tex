\begin{appendices}

\section{Python code}
\label{sec:Appendix_A}

\begin{lstlisting}[frame=single, language=Python, numbers=left]
import math
import pytz
from dateutil.relativedelta import relativedelta
from skyfield.api import load, wgs84, EarthSatellite
import matplotlib
matplotlib.use('QtAgg')
import matplotlib.pyplot as plt
from datetime import timedelta

timescale = load.timescale()

# two-line elements given in homework assignment, updated according to email
line1 = '1 25544U 98067A   24281.53350623  .00033313  00000+0  60150-3 0  9991'
line2 = '2 25544  51.6369 118.9253 0008961  46.9406 313.2331 15.49376493475973'
satellite = EarthSatellite(line1, line2, 'ISS (ZARYA)', timescale)


''' 
Part 1: 
Two lines elements (TLE)
* Orbital elements from TLE-data:
    * read keplerian orbital elements from TLE-data
    * calculate semi-major axis from mean motion and orbital time
    * print values
'''
# Orbital elements
eccentricity = line2[26:33]
inclination = line2[8:16]
right_ascension = line2[17:25]
argument = line2[34:42]
mean_anomaly = line2[43:51]
mean_motion = line2[52:63]

# Calculate semi-major axis
T = 24*60*60/float(mean_motion)
mu = 3.986*10**14
a = (((T**2) * mu) / (4 * (math.pi ** 2)))**(1/3)

# Printing values
print("======== Orbital elements ========")
print("eccentricity: ", eccentricity)
print("inclination: ", inclination)
print("right ascension: ", right_ascension)
print("argument of periapsis: ", argument)
print("mean anomaly: ", mean_anomaly)
print("mean motion: ", mean_motion)

print("Orbital period: ", T, "s or", T/60, "min")
print("Semi-major axis: ", a/1000, "km")
print("==================================")



''' Part 2: 
Ground track computation
* get starting time from epoch
* set timescale and calculate subpoints (latitude and longitude)
* Plot the groundtrack onto an existing map 
    * image credits: https://upload.wikimedia.org/wikipedia/commons/2/23/Blue_Marble_2002.png 
'''
# Get current time in a timezone-aware fashion from the epoch
tz = pytz.timezone('UTC')
dt = satellite.epoch.astimezone(tz)
print()
print(satellite)
print(f"Exectution time: {dt:%Y-%m-%d %H:%M:%S %Z}\n")

# if statement added for ease of use...
if input("Do you want to plot for part 2 [y/n]?") == "y":
    # Split 3 orbits (3*92.94061233 minutes) into 400 evenly spaced Timescales as indicated by points
    # 400 chosen for map visibility reasons, could be 101 ==> one plot every 2 min + endpoints
    orbits_min = 3 * T / 60
    t0 = timescale.utc(dt)
    t1 = timescale.utc(dt + relativedelta(minutes=orbits_min))
    timescales = timescale.linspace(t0, t1, 400)

    # calculate the latitude and longitude subpoints.
    geocentrics = satellite.at(timescales)
    subpoints = wgs84.subpoint_of(geocentrics)
    latitude = subpoints.latitude.degrees
    longitude = subpoints.longitude.degrees

    # Load background image
    background_image_path = r'C:\Users\benja\PycharmProjects\groundtrack\earth.jpg'
    background_img = plt.imread(background_image_path)

    # Create the plot
    plt.figure(figsize=(15.2, 8.2))
    plt.imshow(background_img, extent=[-180, 180, -90, 90])

    # Plot the ground track
    title = f"ISS ground-track from: {dt:%Y-%m-%d %H:%M:%S %Z}"
    plt.scatter(longitude, latitude, label="ISS ground-track", color='red', marker='o', s=1)
    plt.xlabel("Longitude (degrees, \N{DEGREE SIGN})")
    plt.ylabel("Latitude (degrees, \N{DEGREE SIGN})")
    plt.title(title)

    # Show the plot
    plt.legend()
    plt.grid(True, color='w', linestyle=":", alpha=0.4)
    plt.show()



''' Part 4: 
Impact of the J2 parameters
* something
* plot scatter points without J2 RAAN drift in red
* plot scatter points with J2 RAAN drift in blue
'''
if input("Do you want to plot for part 4 [y/n]?") == "y":
    # Split 3 orbits (3*92.94061233 minutes) into 1000 evenly spaced Timescales as indicated by points
    # 400 chosen for map visibility reasons, could be 101 ==> one plot every 2 min + endpoints
    orbits_min = int(10 * T / 60)
    t0 = timescale.utc(dt)
    t1 = timescale.utc(dt + relativedelta(minutes=orbits_min))
    timescales = timescale.linspace(t0, t1, orbits_min+1)

    # calculate the subpoints.
    geocentrics = satellite.at(timescales)
    subpoints = wgs84.subpoint_of(geocentrics)

    '''# Print a nicely-formatted, tab-delimited time, latitude, and longitude.
    for t, lat, lon in zip(timescales,
                           subpoints.latitude.degrees,
                           subpoints.longitude.degrees):
        print(f"{t.astimezone(tz):%Y-%m-%d %H:%M:%S}\t{lat:8.2f}\t{lon:8.2f}")
'''

    # Load background image
    background_image_path = r'C:\Users\benja\PycharmProjects\groundtrack\earth.jpg'
    background_img = plt.imread(background_image_path)
    latitude = subpoints.latitude.degrees
    longitude = subpoints.longitude.degrees

    # Create the plot
    plt.figure(figsize=(15.2, 8.2))
    plt.imshow(background_img, extent=[-180, 180, -90, 90])

    # Plot the ground track with and without J2 RAAN drift (red and blue)
    title = f"ISS ground-track for 10 orbits with and without J2 - from: {dt:%Y-%m-%d %H:%M:%S %Z}"
    plt.xlabel("Longitude (degrees, \N{DEGREE SIGN})")
    plt.ylabel("Latitude (degrees, \N{DEGREE SIGN})")
    plt.title(title)

    # plot without J2 drift
    plt.scatter(longitude, latitude, label="ISS ground-track without J2 RAAN drift", color='red', marker='o', s=1)

    # plot with J2 drift (new longitude and latitude vector
    # calculate the subpoints for changing RAAN and argument of perigee
    # both change every second, so for each e.g. minute we can calculate a new
    # also need a new dt for each iteration...
    # Get current time in a timezone-aware fashion from the epoch
    Omega = str(118.9253)
    omega = str(46.9406)
    for a in range(orbits_min):
        # two-line elements given in homework assignment, updated according to email
        #line1 = '1 25544U 98067A   24281.53350623  .00033313  00000+0  60150-3 0  9991'
        #line2 = '2 25544  51.6369 118.9253 0008961  46.9406 313.2331 15.49376493475973'
        line1 = '1 25544U 98067A   24281.53350623  .00033313  00000+0  60150-3 0  9991'
        line2 = '2 25544  51.6369 {:} 0008961  {:} 313.2331 15.49376493475973'.format(Omega, omega)
        satellite = EarthSatellite(line1, line2, 'ISS (ZARYA)', timescale)

        t0_2 = timescale.utc(dt)
        t1_2 = timescale.utc(dt + relativedelta(minutes=1))
        timescales2 = timescale.linspace(t0_2, t1_2, 1)
        geocentrics2 = satellite.at(timescales2)
        subpoints2 = wgs84.subpoint_of(geocentrics2)
        latitude2 = subpoints2.latitude.degrees
        longitude2 = subpoints2.longitude.degrees
        plt.scatter(longitude2, latitude2, color='blue', marker='o', s=1)

        # update variables
        dt = dt + timedelta(0,60)
        Omega = str(float(Omega) - 0.003437840715)
        omega = str(float(omega) + 0.002564598683)



    # Show the plot
    plt.legend()
    plt.grid(True, color='w', linestyle=":", alpha=0.4)
    plt.show()







\end{lstlisting}


\newpage
\section{Maple calculations}
\label{sec:Maple}
\begin{Maple Normal}
\textbf{M11 Orbital Mechanics II - Homework \#1 }
\end{Maple Normal}
\begin{Maple Normal}
1. Two lines Elements (TLE)
\end{Maple Normal}
\begin{Maple Normal}

\end{Maple Normal}
\begin{Maple Normal}
Orbital parameters:
\end{Maple Normal}
\begin{Maple Normal}
{$ \displaystyle \mathit{ecc} \coloneqq  0.0008961\colon  $}
\end{Maple Normal}
\begin{Maple Normal}
{$ \displaystyle i \coloneqq  51.6369\colon  $}
\end{Maple Normal}
\begin{Maple Normal}
{$ \displaystyle i_{\mathit{rad}}\coloneqq \frac{i \cdot \pi}{180}\colon  $}
\end{Maple Normal}
\begin{Maple Normal}
{$ \displaystyle \Omega \coloneqq  118.9253\colon  $}
\end{Maple Normal}
\begin{Maple Normal}
{$ \displaystyle \mathrm{omega}\coloneqq  46.9406\colon  $}
\end{Maple Normal}
\begin{Maple Normal}

{$ \mathrm{nu}\coloneqq  313.2331\colon  $}
\end{Maple Normal}
\begin{Maple Normal}
{$ \displaystyle n \coloneqq  15.49376493\colon  $}
\end{Maple Normal}
\begin{Maple Normal}

\end{Maple Normal}
\begin{Maple Normal}
Constants:
\end{Maple Normal}
\begin{Maple Normal}
{$ \displaystyle \mu\coloneqq  3.986\cdot 10^{14}\colon  $}
\end{Maple Normal}
\begin{Maple Normal}

\end{Maple Normal}
\begin{Maple Normal}
{$ \displaystyle J_{2}\coloneqq  1.08263\cdot 10^{-3}\colon  $}
\end{Maple Normal}
\begin{Maple Normal}
{$ \displaystyle R_{E}\coloneqq 6378\cdot 10^{3}\colon  $}
\end{Maple Normal}
\begin{Maple Normal}
(at the equator)
\end{Maple Normal}
\begin{Maple Normal}

\end{Maple Normal}
\begin{Maple Normal}
Orbital period in seconds and minutes:
\end{Maple Normal}
\begin{Maple Normal}
{$ \displaystyle T_{s}\coloneqq \frac{24\cdot 60\cdot 60}{n} $}
\end{Maple Normal}
% \mapleresult
\begin{dmath}\label{(1)}
T_{s}\coloneqq  5576.436740
\end{dmath}
\begin{Maple Normal}
{$ \displaystyle T_{\min}\coloneqq \frac{T_{s}}{60} $}
\end{Maple Normal}
% \mapleresult
\begin{dmath}\label{(2)}
T_{\min}\coloneqq  92.94061233
\end{dmath}
\begin{Maple Normal}

\end{Maple Normal}
\begin{Maple Normal}
Semi-major axis calculattions [m]: 
\end{Maple Normal}
\mapleinput
{$ \displaystyle a \coloneqq (\frac{T_{s}^{2}\,\cdot \,\mu}{4\cdot \pi^{2}})^{\mathit{(\frac{1}{3})}} $}

% \mapleresult
\begin{dmath}\label{(3)}
a \coloneqq  6.796683381\times 10^{6}
\end{dmath}
\begin{Maple Normal}
{$ \displaystyle a_{\mathit{km}}=a \cdot 10^{-3} $}
\end{Maple Normal}
% \mapleresult
\begin{dmath}\label{(4)}
a_{\mathit{km}}= 6796.683381
\end{dmath}
\begin{Maple Normal}
Rate of change for 
{$ \Omega \colon  $} and 
{$ \omega \colon  $}
\end{Maple Normal}
\begin{Maple Normal}
{$ \displaystyle \Omega_{\mathit{dot}}\coloneqq -(\frac{3}{2}\cdot \frac{\mathrm{sqrt}(\mathrm{mu})\cdot J_{2}\cdot R_{E}^{2}}{(1-\mathit{ecc}^{2})^{2}\cdot a^{\frac{7}{2}}})\cdot \cos (i_{\mathit{rad}})\cdot \frac{180}{\mathrm{Pi}}\cdot 60 $}
\end{Maple Normal}
% \mapleresult
\begin{dmath}\label{(5)}
\Omega_{\mathit{dot}}\coloneqq - 0.003437840715
\end{dmath}
\begin{Maple Normal}
{$ \displaystyle \omega_{\mathit{dot}}\coloneqq -(\frac{3}{2}\cdot \frac{\mathrm{sqrt}(\mathrm{mu})\cdot J_{2}\cdot R_{E}^{2}}{(1-\mathit{ecc}^{2})^{2}\cdot a^{\frac{7}{2}}})\cdot (\frac{5}{2}\cdot \sin(i_{\mathit{rad}})^{2}-2)\cdot \frac{180}{\mathrm{Pi}}\cdot 60 $}
\end{Maple Normal}
% \mapleresult
\begin{dmath}\label{(6)}
\omega_{\mathit{dot}}\coloneqq  0.002564598683
\end{dmath}
\begin{Maple Normal}

\end{Maple Normal}
\begin{Maple Normal}

\end{Maple Normal}
\begin{Maple Normal}
{$ \displaystyle \frac{10\cdot T_{\min}}{3000} $}
\end{Maple Normal}
% \mapleresult
\begin{dmath}\label{(7)}
 0.3098020411
\end{dmath}
\begin{Maple Normal}

\end{Maple Normal}




\end{appendices}
