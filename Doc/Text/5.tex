\section{Interplanetary transfer $\Delta V$ computation}
\label{sec:interplanetary_transfer}

\textbf{Now let’s take a mission which consists of a heliocentric transfer from the vicinity of Earth to the vicinity of Mars. 
The initial mass, M0=20,000, starts just outside Earth’s sphere of influence (SOI) and finishes just outside Mars’ SOI.
}

\textbf{Consider a two-impulse heliocentric Hohmann transfer and calculate the total $\Delta V$ needed for the mission. 
Assume no staging is involved. 
With a jet velocity of c = 4500 m/s and a specific impulse of I = 459 s (typical for a LOX-PLH rocket).}
\begin{itemize}
    \item[-] \textbf{What is the total delta V needed to arrive near Mars?}
\end{itemize}


\textbf{Additionally, assume that 5\% of the initial mass (1 000 kg) consists of the structure and engine, with the remainder allocated for the payload.}
\begin{itemize}
    \item[-] \textbf{What is the mass of this payload?}
    \item[-] \textbf{What is the duration of the transfer?}
    \item[-] \textbf{Plot this transfer} 
\end{itemize}




Got stuck on J2 propagation plotting and ran out of time...